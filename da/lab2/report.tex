\documentclass[12pt]{article}

\usepackage{fullpage}
\usepackage{multicol,multirow}
\usepackage{tabularx}
\usepackage{ulem}
\usepackage[utf8]{inputenc}
\usepackage[russian]{babel}
\usepackage{pgfplots}
\usepackage{enumitem}
\usepackage{listings}


\begin{document}

\section*{Лабораторная работа №\,2 по курсу дискрeтного анализа: сбалансированные деревья}

Выполнил студент группы 08-208 МАИ \textit{Ширяев Никита}.

\subsection*{Условие}


Реализовать декартово дерево с возможностью поиска, добавления и удаления элементов.

Необходимо создать программную библиотеку, реализующую указанную структуру данных, на основе которой разработать программу-словарь. В словаре каждому ключу, представляющему из себя регистронезависимую последовательность букв английского алфавита длиной не более 256 символов, поставлен в соответствие некоторый номер, от 0 до $2^{64} - 1$. Разным словам может быть поставлен в соответствие один и тот же номер.

Программа должна обрабатывать строки входного файла до его окончания. Каждая строка может иметь следующий формат: \\

• + word 34 — добавить слово «word» с номером 34 в словарь. Программа должна вывести строку «OK», если операция прошла успешно, «Exist», если слово уже находится в словаре.

• - word — удалить слово «word» из словаря. Программа должна вывести «OK», если слово существовало и было удалено, «NoSuchWord», если слово в словаре не было найдено.

• word — найти в словаре слово «word». Программа должна вывести «OK: 34», если слово было найдено; число, которое следует за «OK:» — номер, присвоенный слову при добавле



\subsection*{Метод решения}

Декартово дерево (Treap) — это структура данных, объединяющая в себе бинарное
дерево поиска и кучу. Более строго, это бинарное дерево, в узлах которого хранятся
пары (x,y), где x — это ключ, а y — это приоритет. Основными операциями декартова
дерева является merge и split.
Операция split позволяет сделать следующее: разрезать исходное дерево T по ключу k.
Возвращать она будет такую пару деревьев 〈T1,T2〉, что в дереве T1 ключи меньше k,
а в дереве T2 все остальные: split(T,k)→〈T1,T2〉.
Рассмотрим вторую операцию с декартовыми деревьями — merge. С помощью этой
операции можно слить два декартовых дерева в одно. Причём, все ключи в первом(левом)
дереве должны быть меньше, чем ключи во втором(правом). В результате получается
дерево, в котором есть все ключи из первого и второго деревьев: merge(T1,T2)→T
Используя эти 2 функции, мы можем реализовать основные операции с деревом:\\


• Операция поиска. Используем операцию split два раза: сначала по нашему ключу
x, а потом правое дерево по ключу x+1. Так мы получим три дерева, в первом все
элементы строго меньше x, в третьем строго больше x, а второе дерево может быть
или пустым, или содержать единственный элемент x. Для поиска можно просто
проверить, что второе дерево не пустое, и вывести его значение, в противном
случае будем выводить "NoSuchWord". После этого применяем операцию merge
два раза, чтобы вернуться к исходному дереву. Все остальные операции построены
аналогично. \\
• Операция вставки. Проверяем, что второе дерево пустое, значит такого элемента
пока не существует, а значит, мы можем создать в этом дереве ноду с полученным
от пользователя ключом и значением. В противном случае выводим "Exist".\\
• Операция удаления. Проверяем, что второе дерево не пустое, значит, мы можем
удалить этот элемент, сначала удаляем ноду, а потом указателю на ноду присваиваем
значение null. Если дерево оказалось пустым, то выводим "NoSuchWord".\\


\subsection*{Описание программы}

Для хранения элемента была создана структура, состоящая из двух полей: ключа и значения.

\begin{lstlisting}[language=C++]
char *key;
int priority;
uint64_t value;
node *left, *right
\end{lstlisting}

Также были реализованы функции вставки и удаления, использующие функции $merge$ и $split$.
При доблавлении узла ключу ставится 


\subsection*{Дневник отладки}

\textbf{Ошибка:} Ключи, отличающиеся только регистром, считались разными, что противоречило условию задачи. \\


\textbf{Способ устранения:} Написать функцию \texttt{toLower}, которая будет переводить ключ в нижний регистр перед сравнением. Таким образом, все ключи будут сравниваться без учета регистра.


\subsection*{Выводы}

В данной лабораторной работе было предложено изучить некоторые виды алгоритмов
сбалансированных деревьев. В ходе выполнения было реализовано декартово дерево. Операции
вставки, поиска и удаления выполняются за временную сложность O(logn), где n —
количество элементов. Также были изучены дополнительные операции merge и
split, которые помогают реализовать операции поиска, вставки и удаления.
\end{document}
